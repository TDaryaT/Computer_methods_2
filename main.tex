\documentclass[12pt,a4paper]{scrartcl}
\usepackage[utf8]{inputenc}
\usepackage{mathtools}
\usepackage[english,russian]{babel}
\usepackage{indentfirst}
\usepackage{misccorr}
\usepackage{amsmath}
\usepackage{mathptmx}
\usepackage{physics}

\usepackage{graphicx}

\usepackage[rightcaption]{sidecap}
\usepackage{wrapfig}

\input{for_fin(1)}

\begin{document}

\input{title}

\newpage

\section{Постановка задачи.}

С помощью 5-точечной схемы свести уравнение задачи Дирихле

\[
    -a{\dfrac{{\partial}^2 u}{\partial x^{2}}}-b{\dfrac{{\partial}^2 u}{\partial y^{2}}}=f(x,y), (x,y)\in D 
\]

\[
    u(x,y)= \varphi(x,y), (x,y)\in \partial D
\]

в области $D$ к решению системы линейных алгебраических уравнений $ Az = F $.
Использовать разностную схему с шагами:

\[
    h_x=h_y=h=\frac{1}{m}
\]

\[
    -a\dfrac{U_{i-1,j}-2U_{i,j}+U_{i+1,j}}{h^2}-b\dfrac{U_{i,j-1}-2U_{i,j}+U_{i,j+1}}{h^2}=f_{i,j}
\]

Для

\[
    a=1; b=1
\]

\[
    f(x,y)=2\pi^2sin(\pi x)sin(\pi x),
\]

\[
     \varphi(x,y)=sin(\pi x)sin(\pi x)
\]

Область $D$ выглядит следующим образом:

\begin{figure}[h]
    \centering
    \includegraphics[width=6cm]{img/domain.png}
\end{figure}

\begin{enumerate}
    \item Найти погрешность аппроксимации разностной схемы на решении краевой задачи.
    \item Проверить, что функция $  \varphi(x,y)$ является точным решением краевой задачи.
    \item Для небольшого числа узлов сетки и двух вариантов нумерации внутренних узлов выписать в симметрическом виде матрицы $A$. Убедиться, что они являются симметричными и разреженными.
    \item Записать полученные матрицы в упакованном виде (по строкам).
    \item Найти максимальное и минимальное собственное значения матрицы $A$ для нескольких значений $m$ степенным методом.
    \item Используя собственные значения оператора $(Lv)_i=v_{i-1}-2v_i+v_{i+1},i=1,...,m-1, v_0=v_m=0$, найти собственные значения двумерного разностного оператора рассматриваемой задачи для $D=Q^2=[0,1]^2$. Сравнивать полученное значение $\lambda_{min}(h)$ с $\lambda_{min}$ для дифференциальной задачи. Получить асимптотическое разложение $\lambda_{min}(h)$ по $h$.
    \item Найти решение системы уравнений $Az=F$ методом установления и методом верхней релаксации.
    \item Для нескольких значений числа интервалов $m$ найти относительные погрешности разностного решения $	\Delta=<|U(x,y)-\varphi(x,y)|>/<|\varphi(x,y)|>.$
    
    Здесь $<|g(x,y)|>=\sum_{i,j}[(i,j)=\in D]|g(x_i,y_j)|.$
    \item Вывести на экран разностное решение $U(x,y)$ и погрешность $U(x,y)-\varphi(x,y)$.
    
\end{enumerate}

\section{Исследование данной схемы на точность и устойчивость.}

    \subsection{Погрешность аппроксимации.}
    Оценим погрешность аппроксимации данной задачи, для этого, Приблизим оператор Лапласа разностным оператором:
    \[
        \Lambda u = \Lambda_1 u + \Lambda_2 u = u_{x,x}+u_{y,y}
    \]
    Тогда с помощью разложения в ряд Тейлора получаем:
    \[
        \Lambda_1 u = {\dfrac{{\partial}^2 u}{\partial x^{2}}}+\dfrac{h^2}{12}{\dfrac{{\partial}^4 u}{\partial x^{4}}}+O(h^4)
    \]
    \[
         \Lambda_2 u = {\dfrac{{\partial}^2 u}{\partial y^{2}}}+\dfrac{h^2}{12}{\dfrac{{\partial}^4 u}{\partial y^{4}}}+O(h^4)
    \]
    Тогда
    \[
        \Lambda u - \Delta v = \dfrac{h^2}{12}{\dfrac{{\partial}^4 u}{\partial x^{4}}}+\dfrac{h^2}{12}{\dfrac{{\partial}^4 u}{\partial y^{4}}}+O(h^4)
    \]
    Отсюда следует, что
    \[
        \Lambda u - \Delta v = O(h^2)
    \]
    Таким образом, данный разностный оператор аппроксимирует оператор Лапласа со  вторым порядком аппроксимации на регулярном шаблоне "крест". 
    
\newpage

\section{Описание вычислительного метода. Алгоритм решения программы.}
\begin{itemize}
	\item Изобразим шаблон нашей схемы:
	\begin{figure}[h]
		\centering
		\includegraphics[width=6cm]{img/scheme.jpg}
	\end{figure}
	\item Наложим на нашу область равномерную сетку с шагом $h$     и пронумеруем узлы сетки следующим образом:
	\begin{figure}[h]
		\centering
		\includegraphics[width=6cm]{img/normal.png}
	\end{figure}
	\item С помощью отдельной функции, отделим точки, принадлежащие границе области D, точки, лежащие строго внутри области D и точки не принадлежащие области.  
	\item Для граничных точек сразу вычисляем значения искомой функции
	\item Для всех внутренних точек выполняем разностную схему и ищем их значения, как решение СЛАУ.На основе этого получаем матрицу.
	\item Записываем эту матрицу в упакованном виде
	\item Ищем у полученной матрицы max и min числа
	\item Решаем матрицу (СЛАУ) с помощью метода верхней релаксации и методом установления
	\item Заносим оставшиеся значения в матрицу решений и выводим полученный график
\end{itemize} 

\section{Код программы}


    \begin{tcolorbox}[breakable, size=fbox, boxrule=1pt, pad at break*=1mm,colback=cellbackground, colframe=cellborder]
\prompt{In}{incolor}{1}{\boxspacing}
\begin{Verbatim}[commandchars=\\\{\}]
\PY{c+c1}{\PYZsh{}\PYZpc{}matplotlib inline}
\PY{o}{\PYZpc{}}\PY{k}{matplotlib} notebook
\end{Verbatim}
\end{tcolorbox}

    \begin{tcolorbox}[breakable, size=fbox, boxrule=1pt, pad at break*=1mm,colback=cellbackground, colframe=cellborder]
\prompt{In}{incolor}{2}{\boxspacing}
\begin{Verbatim}[commandchars=\\\{\}]
\PY{k+kn}{import} \PY{n+nn}{math}
\PY{k+kn}{import} \PY{n+nn}{numpy} \PY{k}{as} \PY{n+nn}{np}
\PY{k+kn}{import} \PY{n+nn}{scipy}\PY{n+nn}{.}\PY{n+nn}{linalg} \PY{k}{as} \PY{n+nn}{sla}
\PY{k+kn}{import} \PY{n+nn}{matplotlib}\PY{n+nn}{.}\PY{n+nn}{pyplot} \PY{k}{as} \PY{n+nn}{plt}
\PY{k+kn}{from} \PY{n+nn}{scipy}\PY{n+nn}{.}\PY{n+nn}{sparse} \PY{k+kn}{import} \PY{n}{csc\PYZus{}matrix}
\PY{k+kn}{from} \PY{n+nn}{scipy}\PY{n+nn}{.}\PY{n+nn}{sparse} \PY{k+kn}{import} \PY{n}{csr\PYZus{}matrix}
\PY{k+kn}{from} \PY{n+nn}{scipy} \PY{k+kn}{import} \PY{n}{sparse}
\PY{k+kn}{import} \PY{n+nn}{scipy}
\PY{k+kn}{import} \PY{n+nn}{scipy}\PY{n+nn}{.}\PY{n+nn}{sparse}\PY{n+nn}{.}\PY{n+nn}{linalg} \PY{k}{as} \PY{n+nn}{linalg}

\PY{k+kn}{from} \PY{n+nn}{mpl\PYZus{}toolkits}\PY{n+nn}{.}\PY{n+nn}{mplot3d} \PY{k+kn}{import} \PY{n}{Axes3D}
\PY{k+kn}{from} \PY{n+nn}{matplotlib} \PY{k+kn}{import} \PY{n}{cm}
\PY{k+kn}{from} \PY{n+nn}{matplotlib}\PY{n+nn}{.}\PY{n+nn}{ticker} \PY{k+kn}{import} \PY{n}{LinearLocator}\PY{p}{,} \PY{n}{FormatStrFormatter}
\end{Verbatim}
\end{tcolorbox}

    \begin{tcolorbox}[breakable, size=fbox, boxrule=1pt, pad at break*=1mm,colback=cellbackground, colframe=cellborder]
\prompt{In}{incolor}{3}{\boxspacing}
\begin{Verbatim}[commandchars=\\\{\}]
\PY{n}{a} \PY{o}{=} \PY{l+m+mi}{1}
\PY{n}{b} \PY{o}{=} \PY{l+m+mi}{1}
\PY{n}{m} \PY{o}{=} \PY{l+m+mi}{4}
\PY{n}{h} \PY{o}{=} \PY{l+m+mi}{1} \PY{o}{/} \PY{n}{m}
\PY{n}{eps} \PY{o}{=} \PY{l+m+mf}{1e\PYZhy{}9}
\PY{n}{number} \PY{o}{=} \PY{n}{np}\PY{o}{.}\PY{n}{zeros}\PY{p}{(}\PY{p}{(}\PY{n}{m} \PY{o}{+} \PY{l+m+mi}{1}\PY{p}{,} \PY{n}{m} \PY{o}{+} \PY{l+m+mi}{1}\PY{p}{)}\PY{p}{,} \PY{n}{dtype} \PY{o}{=} \PY{n+nb}{int}\PY{p}{)}
\end{Verbatim}
\end{tcolorbox}

    \begin{tcolorbox}[breakable, size=fbox, boxrule=1pt, pad at break*=1mm,colback=cellbackground, colframe=cellborder]
\prompt{In}{incolor}{4}{\boxspacing}
\begin{Verbatim}[commandchars=\\\{\}]
\PY{k}{def} \PY{n+nf}{check\PYZus{}in}\PY{p}{(}\PY{n}{x}\PY{p}{,} \PY{n}{y}\PY{p}{)}\PY{p}{:} \PY{c+c1}{\PYZsh{}function of checking }
        \PY{k}{return} \PY{p}{(}\PY{p}{(}\PY{p}{(}\PY{o}{\PYZhy{}}\PY{n}{x} \PY{o}{+} \PY{l+m+mf}{0.5}\PY{p}{)} \PY{o}{\PYZlt{}} \PY{n}{y}\PY{p}{)} \PY{o+ow}{and} \PY{p}{(}\PY{p}{(}\PY{n}{x} \PY{o}{+} \PY{l+m+mf}{0.5}\PY{p}{)} \PY{o}{\PYZgt{}} \PY{n}{y}\PY{p}{)} \PY{o+ow}{and}
                \PY{p}{(}\PY{p}{(}\PY{o}{\PYZhy{}}\PY{n}{x} \PY{o}{+} \PY{l+m+mf}{1.5}\PY{p}{)} \PY{o}{\PYZgt{}} \PY{n}{y}\PY{p}{)} \PY{o+ow}{and} \PY{p}{(}\PY{n}{x} \PY{o}{\PYZlt{}} \PY{l+m+mi}{1}\PY{p}{)} \PY{o+ow}{and} \PY{p}{(}\PY{n}{y} \PY{o}{\PYZgt{}} \PY{l+m+mi}{0}\PY{p}{)}\PY{p}{)}
\end{Verbatim}
\end{tcolorbox}

    \begin{tcolorbox}[breakable, size=fbox, boxrule=1pt, pad at break*=1mm,colback=cellbackground, colframe=cellborder]
\prompt{In}{incolor}{5}{\boxspacing}
\begin{Verbatim}[commandchars=\\\{\}]
\PY{k}{def} \PY{n+nf}{check\PYZus{}border}\PY{p}{(}\PY{n}{x}\PY{p}{,}\PY{n}{y}\PY{p}{)}\PY{p}{:}
    \PY{k}{return} \PY{p}{(}\PY{n}{x} \PY{o}{\PYZgt{}}\PY{o}{=} \PY{l+m+mi}{0}\PY{p}{)} \PY{o+ow}{and} \PY{p}{(}\PY{n}{y} \PY{o}{\PYZgt{}}\PY{o}{=} \PY{l+m+mi}{0}\PY{p}{)} \PY{o+ow}{and} \PY{p}{(}\PY{n}{x} \PY{o}{\PYZlt{}}\PY{o}{=} \PY{l+m+mi}{1}\PY{p}{)} \PY{o+ow}{and} \PY{p}{(}\PY{n}{y} \PY{o}{\PYZlt{}}\PY{o}{=} \PY{l+m+mi}{1}\PY{p}{)} \PY{o+ow}{and} \PY{p}{(}\PY{p}{(}\PY{p}{(}\PY{o}{\PYZhy{}}\PY{n}{x} \PY{o}{+} \PY{l+m+mf}{0.5}\PY{p}{)} \PY{o}{==} \PY{n}{y}\PY{p}{)} \PY{o+ow}{or} \PY{p}{(}\PY{p}{(}\PY{n}{x} \PY{o}{+} \PY{l+m+mf}{0.5}\PY{p}{)} \PY{o}{==} \PY{n}{y}\PY{p}{)} \PY{o+ow}{or} \PY{p}{(}\PY{p}{(}\PY{o}{\PYZhy{}}\PY{n}{x} \PY{o}{+} \PY{l+m+mf}{1.5}\PY{p}{)} \PY{o}{==} \PY{n}{y}\PY{p}{)} \PY{o+ow}{or} \PY{p}{(}\PY{p}{(}\PY{n}{x} \PY{o}{==} \PY{l+m+mi}{1}\PY{p}{)} \PY{o+ow}{and} \PY{p}{(}\PY{n}{y} \PY{o}{\PYZlt{}}\PY{o}{=} \PY{l+m+mf}{0.5}\PY{p}{)}\PY{p}{)} \PY{o+ow}{or} \PY{p}{(}\PY{p}{(}\PY{n}{y} \PY{o}{==} \PY{l+m+mi}{0}\PY{p}{)} \PY{o+ow}{and} \PY{p}{(}\PY{n}{x} \PY{o}{\PYZgt{}}\PY{o}{=} \PY{l+m+mf}{0.5}\PY{p}{)}\PY{p}{)}\PY{p}{)}\PY{p}{;}
\end{Verbatim}
\end{tcolorbox}

    \begin{tcolorbox}[breakable, size=fbox, boxrule=1pt, pad at break*=1mm,colback=cellbackground, colframe=cellborder]
\prompt{In}{incolor}{6}{\boxspacing}
\begin{Verbatim}[commandchars=\\\{\}]
\PY{k}{def} \PY{n+nf}{f}\PY{p}{(}\PY{n}{x}\PY{p}{,} \PY{n}{y}\PY{p}{)}\PY{p}{:}
    \PY{k}{return} \PY{l+m+mf}{2.0} \PY{o}{*} \PY{n}{math}\PY{o}{.}\PY{n}{pi} \PY{o}{*} \PY{n}{math}\PY{o}{.}\PY{n}{pi} \PY{o}{*} \PY{n}{math}\PY{o}{.}\PY{n}{sin}\PY{p}{(}\PY{n}{math}\PY{o}{.}\PY{n}{pi} \PY{o}{*} \PY{n}{x}\PY{p}{)} \PY{o}{*} \PY{n}{math}\PY{o}{.}\PY{n}{sin}\PY{p}{(}\PY{n}{math}\PY{o}{.}\PY{n}{pi} \PY{o}{*} \PY{n}{y}\PY{p}{)}

\PY{k}{def} \PY{n+nf}{phi}\PY{p}{(}\PY{n}{x}\PY{p}{,} \PY{n}{y}\PY{p}{)}\PY{p}{:}
    \PY{k}{return} \PY{n}{math}\PY{o}{.}\PY{n}{sin}\PY{p}{(}\PY{n}{math}\PY{o}{.}\PY{n}{pi} \PY{o}{*} \PY{n}{x}\PY{p}{)} \PY{o}{*} \PY{n}{math}\PY{o}{.}\PY{n}{sin}\PY{p}{(}\PY{n}{math}\PY{o}{.}\PY{n}{pi} \PY{o}{*} \PY{n}{y}\PY{p}{)}
\end{Verbatim}
\end{tcolorbox}

    \begin{tcolorbox}[breakable, size=fbox, boxrule=1pt, pad at break*=1mm,colback=cellbackground, colframe=cellborder]
\prompt{In}{incolor}{7}{\boxspacing}
\begin{Verbatim}[commandchars=\\\{\}]
\PY{n}{cnt} \PY{o}{=} \PY{l+m+mi}{0}
\PY{k}{for} \PY{n}{i} \PY{o+ow}{in} \PY{n+nb}{range}\PY{p}{(}\PY{l+m+mi}{1}\PY{p}{,} \PY{n}{m}\PY{p}{)}\PY{p}{:}
    \PY{k}{for} \PY{n}{j} \PY{o+ow}{in} \PY{n+nb}{range}\PY{p}{(}\PY{l+m+mi}{1}\PY{p}{,} \PY{n}{m}\PY{p}{)}\PY{p}{:}
        \PY{k}{if} \PY{n}{check\PYZus{}in}\PY{p}{(}\PY{n}{i} \PY{o}{/} \PY{n}{m}\PY{p}{,} \PY{n}{j} \PY{o}{/} \PY{n}{m}\PY{p}{)}\PY{p}{:}
            \PY{n}{number}\PY{p}{[}\PY{n}{i}\PY{p}{]}\PY{p}{[}\PY{n}{j}\PY{p}{]} \PY{o}{=} \PY{n}{cnt}
            \PY{n}{cnt} \PY{o}{+}\PY{o}{=} \PY{l+m+mi}{1}
\end{Verbatim}
\end{tcolorbox}

    \begin{tcolorbox}[breakable, size=fbox, boxrule=1pt, pad at break*=1mm,colback=cellbackground, colframe=cellborder]
\prompt{In}{incolor}{8}{\boxspacing}
\begin{Verbatim}[commandchars=\\\{\}]
\PY{n}{x} \PY{o}{=} \PY{n}{np}\PY{o}{.}\PY{n}{zeros}\PY{p}{(}\PY{p}{(}\PY{n}{m} \PY{o}{+} \PY{l+m+mi}{1}\PY{p}{,} \PY{n}{m} \PY{o}{+} \PY{l+m+mi}{1}\PY{p}{)}\PY{p}{)}
\PY{n}{y} \PY{o}{=} \PY{n}{np}\PY{o}{.}\PY{n}{zeros}\PY{p}{(}\PY{p}{(}\PY{n}{m} \PY{o}{+} \PY{l+m+mi}{1}\PY{p}{,} \PY{n}{m} \PY{o}{+} \PY{l+m+mi}{1}\PY{p}{)}\PY{p}{)}
\PY{n}{indicate} \PY{o}{=} \PY{n}{np}\PY{o}{.}\PY{n}{zeros}\PY{p}{(}\PY{p}{(}\PY{n}{m} \PY{o}{+} \PY{l+m+mi}{1}\PY{p}{,}\PY{n}{m} \PY{o}{+} \PY{l+m+mi}{1}\PY{p}{)}\PY{p}{,}\PY{n}{dtype} \PY{o}{=} \PY{n+nb}{int}\PY{p}{)}
\PY{n}{x1} \PY{o}{=} \PY{l+m+mi}{0}
\PY{n}{y1} \PY{o}{=} \PY{l+m+mi}{0}
\PY{n}{k} \PY{o}{=} \PY{l+m+mi}{0}

\PY{k}{for} \PY{n}{i} \PY{o+ow}{in} \PY{n+nb}{range}\PY{p}{(}\PY{l+m+mi}{1}\PY{p}{,} \PY{n}{m} \PY{o}{+} \PY{l+m+mi}{1}\PY{p}{)}\PY{p}{:}
    \PY{k}{for} \PY{n}{j} \PY{o+ow}{in} \PY{n+nb}{range}\PY{p}{(}\PY{l+m+mi}{1}\PY{p}{,} \PY{n}{m} \PY{o}{+} \PY{l+m+mi}{1}\PY{p}{)}\PY{p}{:}
        \PY{n}{x}\PY{p}{[}\PY{n}{i}\PY{p}{]}\PY{p}{[}\PY{n}{j}\PY{p}{]} \PY{o}{=} \PY{n}{i} \PY{o}{/} \PY{n}{m}
        \PY{n}{y}\PY{p}{[}\PY{n}{i}\PY{p}{]}\PY{p}{[}\PY{n}{j}\PY{p}{]} \PY{o}{=} \PY{n}{j} \PY{o}{/} \PY{n}{m}
        
\PY{k}{for} \PY{n}{i} \PY{o+ow}{in} \PY{n+nb}{range}\PY{p}{(}\PY{l+m+mi}{0}\PY{p}{,} \PY{n}{m} \PY{o}{+} \PY{l+m+mi}{1}\PY{p}{)}\PY{p}{:}
    \PY{n}{x1}\PY{o}{=}\PY{l+m+mi}{0}
    \PY{k}{for} \PY{n}{j} \PY{o+ow}{in} \PY{n+nb}{range}\PY{p}{(}\PY{l+m+mi}{0}\PY{p}{,} \PY{n}{m} \PY{o}{+} \PY{l+m+mi}{1}\PY{p}{)}\PY{p}{:}
        \PY{k}{if} \PY{p}{(}\PY{n}{check\PYZus{}in}\PY{p}{(}\PY{n}{x1}\PY{p}{,}\PY{n}{y1}\PY{p}{)}\PY{p}{)}\PY{p}{:}
            \PY{n}{indicate}\PY{p}{[}\PY{n}{i}\PY{p}{]}\PY{p}{[}\PY{n}{j}\PY{p}{]} \PY{o}{=} \PY{n}{k}
            \PY{n}{k}\PY{o}{=}\PY{n}{k}\PY{o}{+}\PY{l+m+mi}{1}
        \PY{k}{else} \PY{p}{:}
            \PY{k}{if} \PY{p}{(}\PY{n}{check\PYZus{}border}\PY{p}{(}\PY{n}{x1}\PY{p}{,} \PY{n}{y1}\PY{p}{)}\PY{p}{)}\PY{p}{:}
                \PY{n}{indicate}\PY{p}{[}\PY{n}{i}\PY{p}{]}\PY{p}{[}\PY{n}{j}\PY{p}{]} \PY{o}{=} \PY{o}{\PYZhy{}}\PY{l+m+mi}{1}
            \PY{k}{else} \PY{p}{:}
                \PY{n}{indicate}\PY{p}{[}\PY{n}{i}\PY{p}{]}\PY{p}{[}\PY{n}{j}\PY{p}{]} \PY{o}{=} \PY{o}{\PYZhy{}}\PY{l+m+mi}{2}
        \PY{n}{x1} \PY{o}{=} \PY{n}{x1} \PY{o}{+} \PY{n}{h}           
    \PY{n}{y1} \PY{o}{=} \PY{n}{y1} \PY{o}{+} \PY{n}{h}
\end{Verbatim}
\end{tcolorbox}

    \begin{tcolorbox}[breakable, size=fbox, boxrule=1pt, pad at break*=1mm,colback=cellbackground, colframe=cellborder]
\prompt{In}{incolor}{9}{\boxspacing}
\begin{Verbatim}[commandchars=\\\{\}]
\PY{n}{A} \PY{o}{=} \PY{n}{np}\PY{o}{.}\PY{n}{zeros}\PY{p}{(}\PY{p}{(}\PY{n}{cnt}\PY{p}{,} \PY{n}{cnt}\PY{p}{)}\PY{p}{,} \PY{n}{dtype} \PY{o}{=} \PY{n+nb}{float}\PY{p}{)}
\PY{n}{z} \PY{o}{=} \PY{n}{np}\PY{o}{.}\PY{n}{zeros}\PY{p}{(}\PY{n}{cnt}\PY{p}{)}
\end{Verbatim}
\end{tcolorbox}

    \begin{tcolorbox}[breakable, size=fbox, boxrule=1pt, pad at break*=1mm,colback=cellbackground, colframe=cellborder]
\prompt{In}{incolor}{10}{\boxspacing}
\begin{Verbatim}[commandchars=\\\{\}]
\PY{n}{k} \PY{o}{=} \PY{l+m+mi}{0}
\PY{n}{y1} \PY{o}{=} \PY{n}{h}
\PY{k}{for} \PY{n}{i} \PY{o+ow}{in} \PY{n+nb}{range}\PY{p}{(}\PY{l+m+mi}{1}\PY{p}{,} \PY{n}{m}\PY{p}{)}\PY{p}{:}
    \PY{n}{x1} \PY{o}{=} \PY{n}{h}
    \PY{k}{for} \PY{n}{j} \PY{o+ow}{in} \PY{n+nb}{range}\PY{p}{(}\PY{l+m+mi}{1}\PY{p}{,} \PY{n}{m}\PY{p}{)}\PY{p}{:}
        \PY{k}{if} \PY{p}{(}\PY{n}{indicate}\PY{p}{[}\PY{n}{i}\PY{p}{]}\PY{p}{[}\PY{n}{j}\PY{p}{]} \PY{o}{\PYZgt{}}\PY{o}{=} \PY{l+m+mi}{0}\PY{p}{)} \PY{p}{:}
            \PY{n}{A}\PY{p}{[}\PY{n}{k}\PY{p}{]}\PY{p}{[}\PY{n}{k}\PY{p}{]} \PY{o}{=} \PY{l+m+mi}{2} \PY{o}{*} \PY{p}{(}\PY{n}{a} \PY{o}{+} \PY{n}{b}\PY{p}{)}\PY{o}{/}\PY{p}{(}\PY{n}{h}\PY{o}{*}\PY{n}{h}\PY{p}{)}
            \PY{n}{z}\PY{p}{[}\PY{n}{k}\PY{p}{]} \PY{o}{=}  \PY{n}{f}\PY{p}{(}\PY{n}{x1}\PY{p}{,} \PY{n}{y1}\PY{p}{)}
            \PY{k}{if} \PY{p}{(}\PY{n}{indicate}\PY{p}{[}\PY{n}{i} \PY{o}{\PYZhy{}} \PY{l+m+mi}{1}\PY{p}{]}\PY{p}{[}\PY{n}{j}\PY{p}{]} \PY{o}{\PYZgt{}}\PY{o}{=} \PY{l+m+mi}{0}\PY{p}{)} \PY{p}{:}
                \PY{n}{A}\PY{p}{[}\PY{n}{k}\PY{p}{]}\PY{p}{[}\PY{n}{indicate}\PY{p}{[}\PY{n}{i} \PY{o}{\PYZhy{}} \PY{l+m+mi}{1}\PY{p}{]}\PY{p}{[}\PY{n}{j}\PY{p}{]}\PY{p}{]} \PY{o}{=} \PY{o}{\PYZhy{}}\PY{n}{b}\PY{o}{/}\PY{p}{(}\PY{n}{h}\PY{o}{*}\PY{n}{h}\PY{p}{)}
            \PY{k}{else} \PY{p}{:}
                \PY{k}{if} \PY{p}{(}\PY{n}{indicate}\PY{p}{[}\PY{n}{i} \PY{o}{\PYZhy{}} \PY{l+m+mi}{1}\PY{p}{]}\PY{p}{[}\PY{n}{j}\PY{p}{]} \PY{o}{==} \PY{o}{\PYZhy{}}\PY{l+m+mi}{1}\PY{p}{)} \PY{p}{:}
                    \PY{n}{z}\PY{p}{[}\PY{n}{k}\PY{p}{]} \PY{o}{+}\PY{o}{=} \PY{n}{b} \PY{o}{*} \PY{n}{phi}\PY{p}{(}\PY{n}{x}\PY{p}{[}\PY{n}{i} \PY{o}{\PYZhy{}} \PY{l+m+mi}{1}\PY{p}{]}\PY{p}{[}\PY{n}{j}\PY{p}{]}\PY{p}{,}\PY{n}{y}\PY{p}{[}\PY{n}{i}\PY{o}{\PYZhy{}}\PY{l+m+mi}{1}\PY{p}{]}\PY{p}{[}\PY{n}{j}\PY{p}{]}\PY{p}{)}\PY{o}{/}\PY{p}{(}\PY{n}{h}\PY{o}{*}\PY{n}{h}\PY{p}{)}
                \PY{k}{else} \PY{p}{:}
                    \PY{n+nb}{print}\PY{p}{(}\PY{l+s+s2}{\PYZdq{}}\PY{l+s+s2}{error\PYZus{}1 }\PY{l+s+s2}{\PYZdq{}}\PY{p}{)}
                    
            \PY{k}{if} \PY{p}{(}\PY{n}{indicate}\PY{p}{[}\PY{n}{i}\PY{p}{]}\PY{p}{[}\PY{n}{j} \PY{o}{\PYZhy{}} \PY{l+m+mi}{1}\PY{p}{]} \PY{o}{\PYZgt{}}\PY{o}{=} \PY{l+m+mi}{0}\PY{p}{)} \PY{p}{:}
                \PY{n}{A}\PY{p}{[}\PY{n}{k}\PY{p}{]}\PY{p}{[}\PY{n}{k} \PY{o}{\PYZhy{}} \PY{l+m+mi}{1}\PY{p}{]} \PY{o}{=} \PY{o}{\PYZhy{}}\PY{n}{a}\PY{o}{/}\PY{p}{(}\PY{n}{h}\PY{o}{*}\PY{n}{h}\PY{p}{)}
            \PY{k}{else} \PY{p}{:}    
                \PY{k}{if} \PY{p}{(}\PY{n}{indicate}\PY{p}{[}\PY{n}{i}\PY{p}{]}\PY{p}{[}\PY{n}{j} \PY{o}{\PYZhy{}} \PY{l+m+mi}{1}\PY{p}{]} \PY{o}{==} \PY{o}{\PYZhy{}}\PY{l+m+mi}{1}\PY{p}{)} \PY{p}{:}
                    \PY{n}{z}\PY{p}{[}\PY{n}{k}\PY{p}{]} \PY{o}{+}\PY{o}{=} \PY{n}{a} \PY{o}{*} \PY{n}{phi}\PY{p}{(}\PY{n}{x}\PY{p}{[}\PY{n}{i}\PY{p}{]}\PY{p}{[}\PY{n}{j} \PY{o}{\PYZhy{}} \PY{l+m+mi}{1}\PY{p}{]}\PY{p}{,}\PY{n}{y}\PY{p}{[}\PY{n}{i}\PY{p}{]}\PY{p}{[}\PY{n}{j}\PY{o}{\PYZhy{}}\PY{l+m+mi}{1}\PY{p}{]}\PY{p}{)}\PY{o}{/}\PY{p}{(}\PY{n}{h}\PY{o}{*}\PY{n}{h}\PY{p}{)}
                \PY{k}{else} \PY{p}{:}
                    \PY{n+nb}{print}\PY{p}{(}\PY{l+s+s2}{\PYZdq{}}\PY{l+s+s2}{error\PYZus{}2 }\PY{l+s+s2}{\PYZdq{}}\PY{p}{)}   
                    
            \PY{k}{if} \PY{p}{(}\PY{n}{indicate}\PY{p}{[}\PY{n}{i} \PY{o}{+} \PY{l+m+mi}{1}\PY{p}{]}\PY{p}{[}\PY{n}{j}\PY{p}{]} \PY{o}{\PYZgt{}}\PY{o}{=} \PY{l+m+mi}{0}\PY{p}{)} \PY{p}{:}
                \PY{n}{A}\PY{p}{[}\PY{n}{k}\PY{p}{]}\PY{p}{[}\PY{n}{indicate}\PY{p}{[}\PY{n}{i} \PY{o}{+} \PY{l+m+mi}{1}\PY{p}{]}\PY{p}{[}\PY{n}{j}\PY{p}{]}\PY{p}{]} \PY{o}{=} \PY{o}{\PYZhy{}}\PY{n}{b}\PY{o}{/}\PY{p}{(}\PY{n}{h}\PY{o}{*}\PY{n}{h}\PY{p}{)}
            \PY{k}{else} \PY{p}{:}
                \PY{k}{if} \PY{p}{(}\PY{n}{indicate}\PY{p}{[}\PY{n}{i} \PY{o}{+} \PY{l+m+mi}{1}\PY{p}{]}\PY{p}{[}\PY{n}{j}\PY{p}{]} \PY{o}{==} \PY{o}{\PYZhy{}}\PY{l+m+mi}{1}\PY{p}{)} \PY{p}{:}
                    \PY{n}{z}\PY{p}{[}\PY{n}{k}\PY{p}{]} \PY{o}{+}\PY{o}{=} \PY{n}{b} \PY{o}{*} \PY{n}{phi}\PY{p}{(}\PY{n}{x}\PY{p}{[}\PY{n}{i} \PY{o}{+} \PY{l+m+mi}{1}\PY{p}{]}\PY{p}{[}\PY{n}{j}\PY{p}{]}\PY{p}{,}\PY{n}{y}\PY{p}{[}\PY{n}{i}\PY{o}{+}\PY{l+m+mi}{1}\PY{p}{]}\PY{p}{[}\PY{n}{j}\PY{p}{]}\PY{p}{)}\PY{o}{/}\PY{p}{(}\PY{n}{h}\PY{o}{*}\PY{n}{h}\PY{p}{)}
                \PY{k}{else} \PY{p}{:}
                    \PY{n+nb}{print}\PY{p}{(}\PY{l+s+s2}{\PYZdq{}}\PY{l+s+s2}{error\PYZus{}3 }\PY{l+s+s2}{\PYZdq{}}\PY{p}{)} 
                    
            \PY{k}{if} \PY{p}{(}\PY{n}{indicate}\PY{p}{[}\PY{n}{i}\PY{p}{]}\PY{p}{[}\PY{n}{j} \PY{o}{+} \PY{l+m+mi}{1}\PY{p}{]} \PY{o}{\PYZgt{}}\PY{o}{=} \PY{l+m+mi}{0}\PY{p}{)} \PY{p}{:}
                \PY{n}{A}\PY{p}{[}\PY{n}{k}\PY{p}{]}\PY{p}{[}\PY{n}{k} \PY{o}{+} \PY{l+m+mi}{1}\PY{p}{]} \PY{o}{=} \PY{o}{\PYZhy{}}\PY{n}{a}\PY{o}{/}\PY{p}{(}\PY{n}{h}\PY{o}{*}\PY{n}{h}\PY{p}{)}
            \PY{k}{else} \PY{p}{:}
                \PY{k}{if} \PY{p}{(}\PY{n}{indicate}\PY{p}{[}\PY{n}{i}\PY{p}{]}\PY{p}{[}\PY{n}{j} \PY{o}{+} \PY{l+m+mi}{1}\PY{p}{]} \PY{o}{==} \PY{o}{\PYZhy{}}\PY{l+m+mi}{1}\PY{p}{)} \PY{p}{:}
                    \PY{n}{z}\PY{p}{[}\PY{n}{k}\PY{p}{]} \PY{o}{+}\PY{o}{=} \PY{n}{phi}\PY{p}{(}\PY{n}{x}\PY{p}{[}\PY{n}{i}\PY{p}{]}\PY{p}{[}\PY{n}{j} \PY{o}{+} \PY{l+m+mi}{1}\PY{p}{]}\PY{p}{,}\PY{n}{y}\PY{p}{[}\PY{n}{i}\PY{p}{]}\PY{p}{[}\PY{n}{j}\PY{o}{+}\PY{l+m+mi}{1}\PY{p}{]}\PY{p}{)}\PY{o}{/}\PY{p}{(}\PY{n}{h}\PY{o}{*}\PY{n}{h}\PY{p}{)}
                \PY{k}{else} \PY{p}{:}
                    \PY{n+nb}{print}\PY{p}{(}\PY{l+s+s2}{\PYZdq{}}\PY{l+s+s2}{error\PYZus{}4 }\PY{l+s+s2}{\PYZdq{}}\PY{p}{)}
                    
            \PY{n}{k} \PY{o}{=} \PY{n}{k} \PY{o}{+} \PY{l+m+mi}{1}
        \PY{n}{x1} \PY{o}{=} \PY{n}{x1} \PY{o}{+} \PY{n}{h}
    \PY{n}{y1} \PY{o}{=} \PY{n}{y1} \PY{o}{+} \PY{n}{h}
    
\end{Verbatim}
\end{tcolorbox}

    \begin{tcolorbox}[breakable, size=fbox, boxrule=1pt, pad at break*=1mm,colback=cellbackground, colframe=cellborder]
\prompt{In}{incolor}{11}{\boxspacing}
\begin{Verbatim}[commandchars=\\\{\}]
\PY{k}{def} \PY{n+nf}{find\PYZus{}e}\PY{p}{(}\PY{n}{A}\PY{p}{)}\PY{p}{:} \PY{c+c1}{\PYZsh{} Function to find eigs }
    \PY{n}{size} \PY{o}{=} \PY{n}{A}\PY{o}{.}\PY{n}{shape}\PY{p}{[}\PY{l+m+mi}{0}\PY{p}{]} 
    \PY{n}{y} \PY{o}{=} \PY{n}{np}\PY{o}{.}\PY{n}{array}\PY{p}{(}\PY{p}{[}\PY{l+m+mi}{1} \PY{k}{for} \PY{n}{i} \PY{o+ow}{in} \PY{n+nb}{range}\PY{p}{(}\PY{n}{size}\PY{p}{)}\PY{p}{]}\PY{p}{)} 
    \PY{n}{x} \PY{o}{=} \PY{n}{y} \PY{o}{/} \PY{n}{np}\PY{o}{.}\PY{n}{linalg}\PY{o}{.}\PY{n}{norm}\PY{p}{(}\PY{n}{y}\PY{p}{)} 
    \PY{n}{eps} \PY{o}{=} \PY{l+m+mf}{10e\PYZhy{}6} 
    \PY{n}{lam1} \PY{o}{=} \PY{l+m+mi}{4} 
    \PY{n}{lam2} \PY{o}{=} \PY{l+m+mi}{1} 

    \PY{k}{while} \PY{n+nb}{abs}\PY{p}{(}\PY{n}{lam2} \PY{o}{\PYZhy{}} \PY{n}{lam1}\PY{p}{)} \PY{o}{\PYZgt{}} \PY{n}{eps}\PY{p}{:} 
        \PY{n}{y} \PY{o}{=} \PY{n}{A}\PY{o}{.}\PY{n}{dot}\PY{p}{(}\PY{n}{x}\PY{p}{)} 
        \PY{n}{lam1} \PY{o}{=} \PY{n}{lam2} 
        \PY{n}{lam2} \PY{o}{=} \PY{n}{y}\PY{o}{.}\PY{n}{T}\PY{o}{.}\PY{n}{dot}\PY{p}{(}\PY{n}{x}\PY{p}{)} 
        \PY{n}{x} \PY{o}{=} \PY{n}{y} \PY{o}{/} \PY{n}{np}\PY{o}{.}\PY{n}{linalg}\PY{o}{.}\PY{n}{norm}\PY{p}{(}\PY{n}{y}\PY{p}{)} 

    \PY{n}{eigs} \PY{o}{=} \PY{n}{np}\PY{o}{.}\PY{n}{linalg}\PY{o}{.}\PY{n}{eig}\PY{p}{(}\PY{n}{A}\PY{p}{)}\PY{p}{[}\PY{l+m+mi}{0}\PY{p}{]} 

    \PY{n}{eigs} \PY{o}{=} \PY{n}{np}\PY{o}{.}\PY{n}{sort}\PY{p}{(}\PY{n}{eigs}\PY{p}{)} 
    \PY{n+nb}{print}\PY{p}{(}\PY{l+s+s2}{\PYZdq{}}\PY{l+s+s2}{Eigs: }\PY{l+s+s2}{\PYZdq{}}\PY{p}{,} \PY{n}{eigs}\PY{p}{[}\PY{l+m+mi}{0}\PY{p}{]}\PY{p}{,} \PY{l+s+s2}{\PYZdq{}}\PY{l+s+s2}{ }\PY{l+s+s2}{\PYZdq{}}\PY{p}{,} \PY{n}{eigs}\PY{p}{[}\PY{n}{size} \PY{o}{\PYZhy{}} \PY{l+m+mi}{1}\PY{p}{]}\PY{p}{)}
\end{Verbatim}
\end{tcolorbox}

    \begin{tcolorbox}[breakable, size=fbox, boxrule=1pt, pad at break*=1mm,colback=cellbackground, colframe=cellborder]
\prompt{In}{incolor}{12}{\boxspacing}
\begin{Verbatim}[commandchars=\\\{\}]
\PY{n+nb}{print}\PY{p}{(}\PY{n}{A}\PY{p}{)}
\PY{n}{find\PYZus{}e}\PY{p}{(}\PY{n}{A}\PY{p}{)}
\PY{n}{A} \PY{o}{=} \PY{n}{csr\PYZus{}matrix}\PY{p}{(}\PY{n}{A}\PY{p}{)}
\PY{c+c1}{\PYZsh{}print(A)}
\PY{c+c1}{\PYZsh{}max = scipy.sparse.linalg.eigsh(A,k=5, sigma=0.5)[0]}
\PY{c+c1}{\PYZsh{}print(max)}
\PY{n+nb}{print}\PY{p}{(}\PY{n}{indicate}\PY{p}{)}
\end{Verbatim}
\end{tcolorbox}

    \begin{Verbatim}[commandchars=\\\{\}]
[[ 64. -16.   0. -16.   0.   0.]
 [-16.  64.   0.   0. -16.   0.]
 [  0.   0.  64. -16.   0.   0.]
 [-16.   0. -16.  64. -16. -16.]
 [  0. -16.   0. -16.  64.   0.]
 [  0.   0.   0. -16.   0.  64.]]
Eigs:  27.38807021966818   100.61192978033175
[[-2 -2 -1 -1 -1]
 [-2 -1  0  1 -1]
 [-1  2  3  4 -1]
 [-2 -1  5 -1 -2]
 [-2 -2 -1 -2 -2]]
    \end{Verbatim}

    \begin{tcolorbox}[breakable, size=fbox, boxrule=1pt, pad at break*=1mm,colback=cellbackground, colframe=cellborder]
\prompt{In}{incolor}{13}{\boxspacing}
\begin{Verbatim}[commandchars=\\\{\}]
\PY{n}{u\PYZus{}my} \PY{o}{=} \PY{n}{linalg}\PY{o}{.}\PY{n}{spsolve}\PY{p}{(}\PY{n}{A}\PY{p}{,} \PY{n}{z}\PY{p}{)}
\end{Verbatim}
\end{tcolorbox}

    \begin{tcolorbox}[breakable, size=fbox, boxrule=1pt, pad at break*=1mm,colback=cellbackground, colframe=cellborder]
\prompt{In}{incolor}{14}{\boxspacing}
\begin{Verbatim}[commandchars=\\\{\}]
\PY{n}{u} \PY{o}{=} \PY{n}{np}\PY{o}{.}\PY{n}{zeros}\PY{p}{(}\PY{p}{(}\PY{n}{m} \PY{o}{+} \PY{l+m+mi}{1}\PY{p}{,} \PY{n}{m} \PY{o}{+} \PY{l+m+mi}{1}\PY{p}{)}\PY{p}{,} \PY{n}{dtype}\PY{o}{=}\PY{l+s+s1}{\PYZsq{}}\PY{l+s+s1}{float64}\PY{l+s+s1}{\PYZsq{}}\PY{p}{)}
\PY{k}{for} \PY{n}{i} \PY{o+ow}{in} \PY{n+nb}{range}\PY{p}{(}\PY{n}{m} \PY{o}{+} \PY{l+m+mi}{1}\PY{p}{)}\PY{p}{:}
    \PY{k}{for} \PY{n}{j} \PY{o+ow}{in} \PY{n+nb}{range}\PY{p}{(}\PY{n}{m} \PY{o}{+} \PY{l+m+mi}{1}\PY{p}{)}\PY{p}{:}
        \PY{k}{if} \PY{p}{(}\PY{n}{indicate}\PY{p}{[}\PY{n}{i}\PY{p}{]}\PY{p}{[}\PY{n}{j}\PY{p}{]} \PY{o}{\PYZgt{}}\PY{o}{=} \PY{l+m+mi}{0}\PY{p}{)}\PY{p}{:}
            \PY{n}{u}\PY{p}{[}\PY{n}{i}\PY{p}{]}\PY{p}{[}\PY{n}{j}\PY{p}{]} \PY{o}{=} \PY{n}{u\PYZus{}my}\PY{p}{[}\PY{n}{indicate}\PY{p}{[}\PY{n}{i}\PY{p}{]}\PY{p}{[}\PY{n}{j}\PY{p}{]}\PY{p}{]}
        \PY{k}{else} \PY{p}{:}
            \PY{k}{if} \PY{p}{(}\PY{n}{indicate}\PY{p}{[}\PY{n}{i}\PY{p}{]}\PY{p}{[}\PY{n}{j}\PY{p}{]} \PY{o}{==} \PY{o}{\PYZhy{}}\PY{l+m+mi}{1}\PY{p}{)}\PY{p}{:}
                \PY{n}{u}\PY{p}{[}\PY{n}{i}\PY{p}{]}\PY{p}{[}\PY{n}{j}\PY{p}{]} \PY{o}{=} \PY{n}{phi}\PY{p}{(}\PY{n}{x}\PY{p}{[}\PY{n}{i}\PY{p}{]}\PY{p}{[}\PY{n}{j}\PY{p}{]}\PY{p}{,} \PY{n}{y}\PY{p}{[}\PY{n}{i}\PY{p}{]}\PY{p}{[}\PY{n}{j}\PY{p}{]}\PY{p}{)}
            \PY{k}{else} \PY{p}{:}
                \PY{n}{u}\PY{p}{[}\PY{n}{i}\PY{p}{]}\PY{p}{[}\PY{n}{j}\PY{p}{]} \PY{o}{=} \PY{l+m+mi}{0}
\end{Verbatim}
\end{tcolorbox}

    \begin{tcolorbox}[breakable, size=fbox, boxrule=1pt, pad at break*=1mm,colback=cellbackground, colframe=cellborder]
\prompt{In}{incolor}{15}{\boxspacing}
\begin{Verbatim}[commandchars=\\\{\}]
\PY{n}{u\PYZus{}real} \PY{o}{=} \PY{n}{np}\PY{o}{.}\PY{n}{zeros}\PY{p}{(}\PY{p}{(}\PY{n}{m} \PY{o}{+} \PY{l+m+mi}{1}\PY{p}{,} \PY{n}{m} \PY{o}{+} \PY{l+m+mi}{1}\PY{p}{)}\PY{p}{,} \PY{n}{dtype}\PY{o}{=}\PY{l+s+s1}{\PYZsq{}}\PY{l+s+s1}{float64}\PY{l+s+s1}{\PYZsq{}}\PY{p}{)}
\PY{k}{for} \PY{n}{i} \PY{o+ow}{in} \PY{n+nb}{range}\PY{p}{(}\PY{n}{m} \PY{o}{+} \PY{l+m+mi}{1}\PY{p}{)}\PY{p}{:}
    \PY{k}{for} \PY{n}{j} \PY{o+ow}{in} \PY{n+nb}{range}\PY{p}{(}\PY{n}{m} \PY{o}{+} \PY{l+m+mi}{1}\PY{p}{)}\PY{p}{:}
        \PY{k}{if} \PY{p}{(}\PY{n}{indicate}\PY{p}{[}\PY{n}{i}\PY{p}{]}\PY{p}{[}\PY{n}{j}\PY{p}{]} \PY{o}{\PYZgt{}}\PY{o}{=} \PY{o}{\PYZhy{}}\PY{l+m+mi}{1}\PY{p}{)}\PY{p}{:}
            \PY{n}{u\PYZus{}real}\PY{p}{[}\PY{n}{i}\PY{p}{]}\PY{p}{[}\PY{n}{j}\PY{p}{]} \PY{o}{=} \PY{n}{phi}\PY{p}{(}\PY{n}{x}\PY{p}{[}\PY{n}{i}\PY{p}{]}\PY{p}{[}\PY{n}{j}\PY{p}{]}\PY{p}{,} \PY{n}{y}\PY{p}{[}\PY{n}{i}\PY{p}{]}\PY{p}{[}\PY{n}{j}\PY{p}{]}\PY{p}{)}
        \PY{k}{else} \PY{p}{:}
            \PY{n}{u\PYZus{}real}\PY{p}{[}\PY{n}{i}\PY{p}{]}\PY{p}{[}\PY{n}{j}\PY{p}{]} \PY{o}{=} \PY{l+m+mi}{0}
\end{Verbatim}
\end{tcolorbox}

    \begin{tcolorbox}[breakable, size=fbox, boxrule=1pt, pad at break*=1mm,colback=cellbackground, colframe=cellborder]
\prompt{In}{incolor}{16}{\boxspacing}
\begin{Verbatim}[commandchars=\\\{\}]
\PY{n+nb}{print}\PY{p}{(}\PY{n}{sla}\PY{o}{.}\PY{n}{norm}\PY{p}{(}\PY{n}{u} \PY{o}{\PYZhy{}} \PY{n}{u\PYZus{}real}\PY{p}{)}\PY{p}{)}
\end{Verbatim}
\end{tcolorbox}

    \begin{Verbatim}[commandchars=\\\{\}]
0.06423848246925326
    \end{Verbatim}

    \begin{tcolorbox}[breakable, size=fbox, boxrule=1pt, pad at break*=1mm,colback=cellbackground, colframe=cellborder]
\prompt{In}{incolor}{17}{\boxspacing}
\begin{Verbatim}[commandchars=\\\{\}]
\PY{n}{X}\PY{p}{,} \PY{n}{Y} \PY{o}{=} \PY{n}{np}\PY{o}{.}\PY{n}{meshgrid}\PY{p}{(}\PY{n}{np}\PY{o}{.}\PY{n}{linspace}\PY{p}{(}\PY{l+m+mi}{0}\PY{p}{,} \PY{l+m+mi}{1}\PY{p}{,} \PY{n}{m}  \PY{o}{+} \PY{l+m+mi}{1}\PY{p}{)}\PY{p}{,} \PY{n}{np}\PY{o}{.}\PY{n}{linspace}\PY{p}{(}\PY{l+m+mi}{0}\PY{p}{,} \PY{l+m+mi}{1}\PY{p}{,} \PY{n}{m}  \PY{o}{+} \PY{l+m+mi}{1}\PY{p}{)}\PY{p}{)}

\PY{n}{fig} \PY{o}{=} \PY{n}{plt}\PY{o}{.}\PY{n}{figure}\PY{p}{(}\PY{p}{)}
\PY{n}{ax} \PY{o}{=} \PY{n}{fig}\PY{o}{.}\PY{n}{gca}\PY{p}{(}\PY{n}{projection}\PY{o}{=}\PY{l+s+s1}{\PYZsq{}}\PY{l+s+s1}{3d}\PY{l+s+s1}{\PYZsq{}}\PY{p}{)}
\PY{c+c1}{\PYZsh{}surf = ax.plot\PYZus{}surface(X, Y, u, cmap=cm.coolwarm,}
 \PY{c+c1}{\PYZsh{}                     linewidth=0, antialiased=False)}

\PY{n}{surf\PYZus{}} \PY{o}{=} \PY{n}{ax}\PY{o}{.}\PY{n}{plot\PYZus{}surface}\PY{p}{(}\PY{n}{X}\PY{p}{,} \PY{n}{Y}\PY{p}{,} \PY{n}{u\PYZus{}real}\PY{p}{,} \PY{n}{cmap}\PY{o}{=}\PY{n}{cm}\PY{o}{.}\PY{n}{plasma}\PY{p}{,}
                        \PY{n}{linewidth}\PY{o}{=}\PY{l+m+mi}{0}\PY{p}{,} \PY{n}{antialiased}\PY{o}{=}\PY{k+kc}{False}\PY{p}{)}

\PY{n}{ax}\PY{o}{.}\PY{n}{set\PYZus{}xlabel}\PY{p}{(}\PY{l+s+s1}{\PYZsq{}}\PY{l+s+s1}{t}\PY{l+s+s1}{\PYZsq{}}\PY{p}{)}
\PY{n}{ax}\PY{o}{.}\PY{n}{set\PYZus{}ylabel}\PY{p}{(}\PY{l+s+s1}{\PYZsq{}}\PY{l+s+s1}{x}\PY{l+s+s1}{\PYZsq{}}\PY{p}{)}
\PY{n}{ax}\PY{o}{.}\PY{n}{set\PYZus{}zlabel}\PY{p}{(}\PY{l+s+s1}{\PYZsq{}}\PY{l+s+s1}{u}\PY{l+s+s1}{\PYZsq{}}\PY{p}{)}

\PY{c+c1}{\PYZsh{}fig.colorbar(surf, shrink=0.5, aspect=5)}
\PY{n}{fig}\PY{o}{.}\PY{n}{colorbar}\PY{p}{(}\PY{n}{surf\PYZus{}}\PY{p}{,} \PY{n}{shrink}\PY{o}{=}\PY{l+m+mf}{0.5}\PY{p}{,} \PY{n}{aspect}\PY{o}{=}\PY{l+m+mi}{5}\PY{p}{)}

\PY{n}{plt}\PY{o}{.}\PY{n}{show}\PY{p}{(}\PY{p}{)}
\end{Verbatim}
\end{tcolorbox}

    
    \begin{verbatim}
<IPython.core.display.Javascript object>
    \end{verbatim}

    
    
    \begin{verbatim}
<IPython.core.display.HTML object>
    \end{verbatim}

    
    \begin{tcolorbox}[breakable, size=fbox, boxrule=1pt, pad at break*=1mm,colback=cellbackground, colframe=cellborder]
\prompt{In}{incolor}{18}{\boxspacing}
\begin{Verbatim}[commandchars=\\\{\}]
\PY{n}{X}\PY{p}{,} \PY{n}{Y} \PY{o}{=} \PY{n}{np}\PY{o}{.}\PY{n}{meshgrid}\PY{p}{(}\PY{n}{np}\PY{o}{.}\PY{n}{linspace}\PY{p}{(}\PY{l+m+mi}{0}\PY{p}{,} \PY{l+m+mi}{1}\PY{p}{,} \PY{n}{m}  \PY{o}{+} \PY{l+m+mi}{1}\PY{p}{)}\PY{p}{,} \PY{n}{np}\PY{o}{.}\PY{n}{linspace}\PY{p}{(}\PY{l+m+mi}{0}\PY{p}{,} \PY{l+m+mi}{1}\PY{p}{,} \PY{n}{m}  \PY{o}{+} \PY{l+m+mi}{1}\PY{p}{)}\PY{p}{)}

\PY{n+nb}{print}\PY{p}{(}\PY{n}{X}\PY{o}{.}\PY{n}{shape}\PY{p}{,} \PY{n}{Y}\PY{o}{.}\PY{n}{shape}\PY{p}{)}

\PY{n}{fig} \PY{o}{=} \PY{n}{plt}\PY{o}{.}\PY{n}{figure}\PY{p}{(}\PY{p}{)}
\PY{n}{ax} \PY{o}{=} \PY{n}{fig}\PY{o}{.}\PY{n}{gca}\PY{p}{(}\PY{n}{projection}\PY{o}{=}\PY{l+s+s1}{\PYZsq{}}\PY{l+s+s1}{3d}\PY{l+s+s1}{\PYZsq{}}\PY{p}{)}
\PY{n}{surf} \PY{o}{=} \PY{n}{ax}\PY{o}{.}\PY{n}{plot\PYZus{}surface}\PY{p}{(}\PY{n}{X}\PY{p}{,} \PY{n}{Y}\PY{p}{,} \PY{n}{u} \PY{o}{\PYZhy{}} \PY{n}{u\PYZus{}real}\PY{p}{,} \PY{n}{cmap}\PY{o}{=}\PY{n}{cm}\PY{o}{.}\PY{n}{coolwarm}\PY{p}{,}
                       \PY{n}{linewidth}\PY{o}{=}\PY{l+m+mi}{0}\PY{p}{,} \PY{n}{antialiased}\PY{o}{=}\PY{k+kc}{False}\PY{p}{)}


\PY{n}{ax}\PY{o}{.}\PY{n}{set\PYZus{}xlabel}\PY{p}{(}\PY{l+s+s1}{\PYZsq{}}\PY{l+s+s1}{t}\PY{l+s+s1}{\PYZsq{}}\PY{p}{)}
\PY{n}{ax}\PY{o}{.}\PY{n}{set\PYZus{}ylabel}\PY{p}{(}\PY{l+s+s1}{\PYZsq{}}\PY{l+s+s1}{x}\PY{l+s+s1}{\PYZsq{}}\PY{p}{)}
\PY{n}{ax}\PY{o}{.}\PY{n}{set\PYZus{}zlabel}\PY{p}{(}\PY{l+s+s1}{\PYZsq{}}\PY{l+s+s1}{u}\PY{l+s+s1}{\PYZsq{}}\PY{p}{)}

\PY{n}{fig}\PY{o}{.}\PY{n}{colorbar}\PY{p}{(}\PY{n}{surf}\PY{p}{,} \PY{n}{shrink}\PY{o}{=}\PY{l+m+mf}{0.5}\PY{p}{,} \PY{n}{aspect}\PY{o}{=}\PY{l+m+mi}{5}\PY{p}{)}

\PY{n}{plt}\PY{o}{.}\PY{n}{show}\PY{p}{(}\PY{p}{)}
\end{Verbatim}
\end{tcolorbox}

    \begin{Verbatim}[commandchars=\\\{\}]
(5, 5) (5, 5)
    \end{Verbatim}

    
    \begin{verbatim}
<IPython.core.display.Javascript object>
    \end{verbatim}

    
    
    \begin{verbatim}
<IPython.core.display.HTML object>
    \end{verbatim}

    

    % Add a bibliography block to the postdoc
    



\section{Графический вывод (Тесты)}
При $h = 0.25$ \\
Функция решения и действительная функция:
\begin{figure}[h]
	\begin{minipage}[h]{0.49\linewidth}
		\center{\includegraphics[width=1\linewidth]{img/u_real4.png} \\ а)}
	\end{minipage}
	\hfill
	\begin{minipage}[h]{0.49\linewidth}
		\center{\includegraphics[width=1\linewidth]{img/u4.png} \\ б)}
	\end{minipage}
	\caption{a)решение системы, б)действительное значение при $n=4$}
	\label{ris:image1}
\end{figure}
\\
Значения минимального и максимального собственных чисел: 
\[
	Eigs:  27.38807021966818 \;\;100.61192978033175
\]
Ошибка выглядит следующим образом:
\\
\begin{figure}[h]
	\center{\includegraphics[width=0.5\linewidth]{img/error4.png}}
	\caption{ошибка при $n=4$}
	\label{ris:image}
\end{figure}
\newpage
При $h = 0.625$:
\\
Функция решения и действительная функция:
\\
\begin{figure}[h]
	\begin{minipage}[h]{0.49\linewidth}
		\center{\includegraphics[width=1\linewidth]{img/u_real16.png} \\ а)}
	\end{minipage}
	\hfill
	\begin{minipage}[h]{0.49\linewidth}
		\center{\includegraphics[width=1\linewidth]{img/u16.png} \\ б)}
	\end{minipage}
	\caption{a)решение системы, б)действительное значение при $n=16$}
	\label{ris:image1}
\end{figure}
\\
Значения минимального и максимального собственных чисел: 
\[
	Eigs:  31.841084906917786\;\;2016.1589150930836
\]
Ошибка выглядит следующим образом:
\\
\begin{figure}[h]
	\center{\includegraphics[width=0.5\linewidth]{img/error16.png}}
	\caption{ошибка при $n=16$}
	\label{ris:image}
\end{figure}
\\
Заметим, что во втором тесте мы увеличили $h$ в 4 раза и наша ошибка уменьшилась в 17,5 раз, даже лучше чем должна была (теоретически должна была уменьшиться в 16 раз).

\newpage
\section{Выводы.}

Таким образом мы исследовали задачу Дирихле в ограниченной области и изучили ход ее решения. Убедились в том, что при увеличении шага $h$, погрешность решения уменьшается примерно в $h^2$ раз.

Этот метод не очень прост в реализации: в нем участвуют сразу несколько дополнительных методов для поиска собственных чисел и решения системы линейных уравнений. Мы научились работать с упаковочными массивами, графически изображать решение и итерационно решать дифференциальные уравнения.

Таким образом,данный метод можно разными способами модифицировать, что также является плюсом метода.

\end{document}
