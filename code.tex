	\begin{verbatim}
	#%matplotlib inline
	get_ipython().run_line_magic('matplotlib', 'notebook')
	
	import math
	import numpy as np
	import scipy.linalg as sla
	import matplotlib.pyplot as plt
	from scipy.sparse import csc_matrix
	from scipy.sparse import csr_matrix
	from scipy import sparse
	import scipy
	import scipy.sparse.linalg as linalg
	
	from mpl_toolkits.mplot3d import Axes3D
	from matplotlib import cm
	from matplotlib.ticker import LinearLocator, FormatStrFormatter
	
	#main parametrs
	
	a = 1
	b = 1
	m = 4
	h = 1 / m
	eps = 1e-9
	number = np.zeros((m + 1, m + 1), dtype = int)
	
	#internal point check function
	
	def check_in(x, y): 
	return (((-x + 0.5) < y) and ((x + 0.5) > y) and
	((-x + 1.5) > y) and (x < 1) and (y > 0))
	
	# boundary point check function
	
	def check_border(x,y):
	return (x >= 0) and (y >= 0) and (x <= 1) and (y <= 1) and (((-x + 0.5) == y) or ((x + 0.5) == y) or ((-x + 1.5) == y) or ((x == 1) and (y <= 0.5)) or ((y == 0) and (x >= 0.5)));
	
	# main and boundary function
	
	def f(x, y):
	return 2.0 * math.pi * math.pi * math.sin(math.pi * x) * math.sin(math.pi * y)
	
	def phi(x, y):
	return math.sin(math.pi * x) * math.sin(math.pi * y)
	
	#
	
	cnt = 0
	for i in range(1, m):
	for j in range(1, m):
	if check_in(i / m, j / m):
	number[i][j] = cnt
	cnt += 1
	
	#filling the array of indicators
	
	x = np.zeros((m + 1, m + 1))
	y = np.zeros((m + 1, m + 1))
	indicate = np.zeros((m + 1,m + 1),dtype = int)
	x1 = 0
	y1 = 0
	k = 0
	
	for i in range(1, m + 1):
	for j in range(1, m + 1):
	x[i][j] = i / m
	y[i][j] = j / m
	
	for i in range(0, m + 1):
	x1=0
	for j in range(0, m + 1):
	if (check_in(x1,y1)):
	indicate[i][j] = k
	k=k+1
	else :
	if (check_border(x1, y1)):
	indicate[i][j] = -1
	else :
	indicate[i][j] = -2
	x1 = x1 + h           
	y1 = y1 + h
	
	
	A = np.zeros((cnt, cnt), dtype = float)
	z = np.zeros(cnt)
	
	# filling the array of scheme
	
	k = 0
	y1 = h
	for i in range(1, m):
	x1 = h
	for j in range(1, m):
	if (indicate[i][j] >= 0) :
	A[k][k] = 2 * (a + b)/(h*h)
	z[k] =  f(x1, y1)
	if (indicate[i - 1][j] >= 0) :
	A[k][indicate[i - 1][j]] = -b/(h*h)
	else :
	if (indicate[i - 1][j] == -1) :
	z[k] += b * phi(x[i - 1][j],y[i-1][j])/(h*h)
	else :
	print("error_1 ")
	
	if (indicate[i][j - 1] >= 0) :
	A[k][k - 1] = -a/(h*h)
	else :    
	if (indicate[i][j - 1] == -1) :
	z[k] += a * phi(x[i][j - 1],y[i][j-1])/(h*h)
	else :
	print("error_2 ")   
	
	if (indicate[i + 1][j] >= 0) :
	A[k][indicate[i + 1][j]] = -b/(h*h)
	else :
	if (indicate[i + 1][j] == -1) :
	z[k] += b * phi(x[i + 1][j],y[i+1][j])/(h*h)
	else :
	print("error_3 ") 
	
	if (indicate[i][j + 1] >= 0) :
	A[k][k + 1] = -a/(h*h)
	else :
	if (indicate[i][j + 1] == -1) :
	z[k] += phi(x[i][j + 1],y[i][j+1])/(h*h)
	else :
	print("error_4 ")
	
	k = k + 1
	x1 = x1 + h
	y1 = y1 + h
	
	# Function to find eigs 
	
	def find_e(A): 
	size = A.shape[0] 
	y = np.array([1 for i in range(size)]) 
	x = y / np.linalg.norm(y) 
	eps = 10e-6 
	lam1 = 4 
	lam2 = 1 
	
	while abs(lam2 - lam1) > eps: 
	y = A.dot(x) 
	lam1 = lam2 
	lam2 = y.T.dot(x) 
	x = y / np.linalg.norm(y) 
	
	eigs = np.linalg.eig(A)[0] 
	
	eigs = np.sort(eigs) 
	print("Eigs: ", eigs[0], " ", eigs[size - 1])
	
	# finding the max and min eigenvalues of scheme array
	
	print(A)
	find_e(A)
	A = csr_matrix(A)
	#print(A)
	#max = scipy.sparse.linalg.eigsh(A,k=5, sigma=0.5)[0]
	#print(max)
	print(indicate)
	
	# solving a system of equations
	
	u_my = linalg.spsolve(A, z)
	
	# filling the solution array
	
	u = np.zeros((m + 1, m + 1), dtype='float64')
	for i in range(m + 1):
	for j in range(m + 1):
	if (indicate[i][j] >= 0):
	u[i][j] = u_my[indicate[i][j]]
	else :
	if (indicate[i][j] == -1):
	u[i][j] = phi(x[i][j], y[i][j])
	else :
	u[i][j] = 0
	
	# filling the array of valid values
	
	u_real = np.zeros((m + 1, m + 1), dtype='float64')
	for i in range(m + 1):
	for j in range(m + 1):
	if (indicate[i][j] >= -1):
	u_real[i][j] = phi(x[i][j], y[i][j])
	else :
	u_real[i][j] = 0
	
	# plotting the solution array
	
	X, Y = np.meshgrid(np.linspace(0, 1, m  + 1), np.linspace(0, 1, m  + 1))
	
	fig = plt.figure()
	ax = fig.gca(projection='3d')
	#surf = ax.plot_surface(X, Y, u, cmap=cm.coolwarm,
	linewidth=0, antialiased=False)
	
	surf_ = ax.plot_surface(X, Y, u_real, cmap=cm.plasma,
	linewidth=0, antialiased=False)
	
	ax.set_xlabel('t')
	ax.set_ylabel('x')
	ax.set_zlabel('u')
	
	#fig.colorbar(surf, shrink=0.5, aspect=5)
	fig.colorbar(surf_, shrink=0.5, aspect=5)
	
	plt.show()
	
	# plotting the array of valid values
	
	X, Y = np.meshgrid(np.linspace(0, 1, m  + 1), np.linspace(0, 1, m  + 1))
	
	fig = plt.figure()
	ax = fig.gca(projection='3d')
	surf = ax.plot_surface(X, Y, u, cmap=cm.coolwarm,
	linewidth=0, antialiased=False)
	
	# surf_ = ax.plot_surface(X, Y, u_real, cmap=cm.plasma,
	#                       linewidth=0, antialiased=False)
	
	ax.set_xlabel('t')
	ax.set_ylabel('x')
	ax.set_zlabel('u')
	
	fig.colorbar(surf, shrink=0.5, aspect=5)
	# fig.colorbar(surf_, shrink=0.5, aspect=5)
	
	plt.show()
	
	# plotting the error
	
	X, Y = np.meshgrid(np.linspace(0, 1, m  + 1), np.linspace(0, 1, m  + 1))
	
	print(X.shape, Y.shape)
	
	fig = plt.figure()
	ax = fig.gca(projection='3d')
	surf = ax.plot_surface(X, Y, u - u_real, cmap=cm.coolwarm,
	linewidth=0, antialiased=False)
	
	
	ax.set_xlabel('t')
	ax.set_ylabel('x')
	ax.set_zlabel('u')
	
	fig.colorbar(surf, shrink=0.5, aspect=5)
	
	plt.show()

	\end{verbatim}
	